\magnification=1200

\parindent=0pt
\parskip=\smallskipamount

\def\next{{\it next slide}}

Dear colleagues and friends, thank you all for coming today, and
greetings to colleagues who join us online.  And thank you to the
citizens of Kanton Zuerich, for making the university possible.  The
University has done many fine things since it was set up in 1833, but
perhaps the most far-reaching of all was to change what a university
is.  As we know, it was the ordinary people who decided in the 1830s
to have a university here.  A crazy idea at the time. Bologna, Basel,
Heidelberg, Paris, Oxbridge would have laughed at this upstart -- but
look at them today, is any of them not a public university?

That said, relations with the public have often been colourful.  I
tried looking up some of the early professors, whose statues are
benevolently looking upon us.  But I found the really interesting
professor isn't one of the statues. David Strauss was appointed a
professor in 1839 of dogmatic theology -- you can't make this stuff
up.  But apparently he wasn't dogmatic enough, and went around saying
all that some of the life of Jesus should be read as mythology.  And
the public were so enraged that he was pensioned off before he could
start.  And here he is, the devilish preacher on his Strauss, his
ostrich.

The history of light and gravity, however, goes back even further.

\next

Here's Newton, posing a question\dots Now in 1704, Newton is
recognized by his contemporaries as the lion among them, even by the
ones who don't like him.  But he is in his sixties, and though he is
still very active, he probably realizes he is not going to solve
everything he wants to in one lifetime.  And he writes this book on
light, called Opticks, and at the end he puts these Queries, sort of
like problems for future generations.

Future generations did take it seriously.  The one most remembered now
is Soldner, a century later.  Soldner knew that fast moving particles
near the Sun don't go into orbit, they only get slightly deflected.
And he calculated how much something moving at the speed of light
would be deflected in Newtonian gravity, and says, look it's very
small but measurable.  The idea is: pick a star, like Regulus in Leo,
which goes behind the Sun.  First take good observations of Leo during
winter nights.  Then observe Leo in August, during the day, when
Regulus is at the rim of the Sun.  Regulus should appear slightly
displaced, relative to rest of Leo, because of light deflection.
Soldner is so close to doing the experiment in 1801.  But then he
gives up, saying you'll never see stars at the rim of the Sun.  You
read his paper now, and you're like -- man, haven't you heard of
eclipses?  But nobody picks up on this.  Through the nineteenth
century, physicists gradually unravel electricity and magnetism.
Light is a wave, of alternating electric and magnetic fields, why
should gravity do anything to it?

Until eventually, a certain Albert Einstein finds the answer to
Newton's question.  Yes, he said, light is deflected by gravity, but
twice as much as you'd guess in Newton's theory.  And by the way, he
said: light is both a particle and a wave.

\next

In 1919 Eddington and friends finally did the experiment Soldner never
did.  And here's their result: Newtonian theory boo, Einsteinian
theory yay.  That's how the newspapers presented it, and Einstein
became a household name.

\next

It would be sixty years before light deflection made the mass media
again.  In the 1970s the forerunners of today's digital cameras got
put on telescopes, and astronomers could image much much fainter
things.  The images weren't so pretty yet -- these are from the Hubble
telescope much later -- but the essentials could be seen.  This is a
light-deflection experiment on a much much larger scale.  Instead of
the Sun, we have a galaxy maybe a billion light years away.  Instead
of a star we have a quasar several billion light-years away.  The
gravity of the whole galaxy pulls the light so much that some of it
comes out the other side.  We can see two images of the quasar, or
even four.  These are now called gravitational lenses.  They are not
common, because you need a bright thing like a quasar almost directly
behind a massive thing like a galaxy.  But a few hundred have been
found scattered around the sky.  These were the first two.  There's
another thing.  The quasar here varies in intensity.  So sometimes
this image varies in intensity over a period of days.  Then about a
year later, well 59 weeks later, the other image does the same.  The
light travel time is billions of years, but this one is 59 weeks
longer.  But only a few percent of lenses does it happen that the
source is so conveniently variable.

To understand these mirages, we really need to delve into Einstein's
theory a little bit.  And the central idea is actually not unfamiliar
to us.

\next

Now Hawaii is south of here, right?  But to fly there, the shortest
route takes you over Greenland.

\next

And Cape Town to Auckland takes you over Antarctica.  These
flight-path curves are called geodesics.  It's easy to move along a
geodesic: you just go in some direction and keep going.  These South
Africans on their way to New Zealand will see themselves as flying
straight.  It's only the map that tells them they go SE first and then
NE.  All because the Earth's surface is curved.

Now, if you're in orbit or just coasting through space, you don't feel
any gravitational force pull you.  Like those South Africans on the
plane, you seem to be just moving in a straight line.  It's only the
outside world that thinks you are being pulled by something.  Einstein
thought a lot about scenarios like this, and then said: maybe gravity
isn't really a force at all, maybe it's space and time getting curved.
And he called it general relativity.

In general relativity, both space and time can get curved.  We mostly
only notice the curvature in time, because we don't move very much in
space.  And Newtonian gravity comes out as an approximate theory of
time curvature.  But light moves a lot in space, so it notices space
curvature too.  That's why gravity is stronger for light in Einstein's
theory than in Newton's.

So Newton posed the question of light and gravity and Einstein
answered it.  Is there really anything more to say?  Yes, we haven't
heard from Darwin yet.

\next

What on Earth is Charles Darwin doing, writing about gravity, a
hundred years after the Origin of Species?!  Well, this is Charles
Darwin's grandson, a very fine mathematical physicist. This paper
could have been called light near a black hole, but the term black
hole didn't exist yet.

\next

Here's Ray Angelil's animation.

A slightly technical point: is it valid to represent photons as dots?
What about interference?  Well, I think it's ok to represent photons
as dots if the observer's time resolution is much coarser than the
coherence times of the light, and that is the regime we expect to be
in, in the foreseeable future.  One can calculate interference and
quantum-optical effects: the most fascinating is Hawking radiation,
which is a black hole glowing very very faintly.  But we don't have an
idea of where to begin to look for something like that, so I'll leave
that topic, regretfully.

\next

Now let's look at another scenario.  This time, we start a bunch of
photons some way away from the black hole.  Some will fall in: I've
omitted those from the animation altogether, leaving this gap.  The
others will travel past the black hole, getting deflected.

\next

Let's add more photons, so it's like setting of a light flash.  The
observer somewhere on the rim will see the photons coming from two
different directions, and then see the whole thing again, but fainter.

Seeing that echo would be cool, but it's not really feasible yet.  The
situation one can observe is more like this:

\next

You have some mass, not necessarily a black hole, and you can see two
images.  If the mass isn't round, you may see more images.

\next

The next step is going from a general understanding of why you get
multiple images to actually modelling the observations.  That is say,
figuring out what the gravitational field is, and what kind of mass
distribution in these lensing galaxies would produce it.  I'd like to
explain one way of going about that, not the only way, but a very
intuitive one.  That's using Fermat's principle.  Fermat's principle,
in modern form, is that light is like an airline company: it wants to
get from a light source to an observer, and it finds a geodesic to
take it.  If there's more than one geodesic, it takes them all.

\next

Let's go back to those light-flash animations and do a thought
experiment --- you couldn't really do this, it's just useful to think
about.  Let's freeze this blue flash, and send this yellow light flash
out from the observer.  When the yellow crosses the blue, we draw a
magenta dot here.  The horizontal axis labels the yellow flash
according to the initial outward angle.  The vertical axis is the time
the yellow crosses the blue.  Now, let's imagine the blue photons,
instead of freezing, carry on in the direction the yellow was coming
from.  The right figure would be the arrival time as a function of
direction.  It turns out, this magenta thing is straightforward to
calculate.  But it's not a physical quantity.  Because photons are not
going to jump from one flash to the other.  Except, when a yellow
comes from the direction the blue was going in anyway.  And when is
that?  When blue and yellow are tangential.  Which is when this
magenta becomes horizontal.  These horizontal points represent when
the light flash arrives, and from which direction.  The rest is just
scaffolding.

This whole animation is, of course, just a slice.  We need to rotate
the blue about the horizontal axis, and the yellow about another axis.
And this magenta will become a surface, with a minimum here, a maximum
here, and this one will actually be saddle point.

\next

OK, let's get more concrete.  This is Irchel campus, with North to the
right.  What's the topography?  The lowest point will be, say, at the
pond here.  The highest point will be among the buildings.  What will
the elevation contours look like?  Well, there will be one contour
that crosses itself, and the crossing point will be a saddle point.
From the saddle point, the elevation goes up along this and down along
this.

\next

It can get more interesting.  The pond is actually a double pond, with
a kind of isthmus in between.  So it's two minima with a saddle in
between.  The nice thing is that you can characterize any topographic
map using just these two kinds of contours.  In particular, the
arrival-time surface for photons.

\next

So here's that double-image quasar, the first gravitational lens to be
discovered.  Here's the arrival-time surface, according to a detailed
model.  Remember the surface is just an artificial construct -- we can
only observe images at the minimum here and the saddle-point here.
Qualitatively, it's rather like Irchel campus.

The first four-image system, we have two minima and two saddle points.
The topography of the Irchel campus doesn't look so similar, but if
you imagine the pond stretching down to the left, it would be.

Once you have this modelling machinery in hand, you can go on to some
very interesting things.

\next

One is study dark matter.  What's dark matter?  Well, here's a galaxy.
Not special as galaxies go, just a pretty face.  The collage on the
left is from Astronomy Pictures of the Day and it has several false
colour images: in infrared, visible light, and x-rays, and a
composite.  On the right, is a picture in visible light, but with a
much longer exposure.  The pretty face is completely overexposed and
here's it's replaced by a negative.  And what we can see now is this
very faint kind of halo of stars.  These halo stars are presumably in
orbit, in a gravitational field.  But what's generating that
gravitational field?  It can't be the stars themselves, there's not
enough mass in stars to do that -- otherwise the halo would have been
much brighter.  There has to be a lot of mass there, which we can't
see.  There are several things it could be, but nobody really knows
yet, and meanwhile we just unimaginatively call it dark matter.

\next

You can use gravitational lensing to map the dark halos around
galaxies.  This animation is by Dominik's Leier -- summarizing his PhD
thesis in six seconds, so I wouldn't try to read everything now -- but
what Dominik and Ignacio Ferreras have done here is to step outwards
through a sample of lensing galaxies and plot the mass in stars inside
against the total mass inside.  We see that at first the total mass
tracks the stellar mass, but then the total mass seems to break away
and increase more than the stellar mass.

Another thing you can do with gravitational lenses is study the
expanding universe.

\next

By the way, if you like non-technical science books, try this one by
two of our ETH colleagues.  I read it last summer and really learned a
lot.

\next

Here's an animation of the expanding universe.  Actually, of two
universes.  On the left we have what was thought --until 1998-- to be
our universe: it's called the Einstein de Sitter model.  We see it
from a little after the Big Bang, expanding very fast at first, and
then slowing down.  The animation pauses for a couple of seconds at
the present --- the pale blue dot is us --- and then continues into
the future.  On the right, we have what it now thought to be our
Universe, and is full of red wine.  It's usually called dark energy,
which is a terribly misleading name, because it's not really energy at
all.  But since nobody knows what it is anyway, and one of the
discoverers, Brian Schmidt, grows Pinot Noir in his spare time -- it
might as well be Pinot Noir.  And what this Pinot Noir does, is stop
the deceleration of the universe and make it accelerate again.

\next

Now let's have some fireworks go off in the universe.  These
explosions are completely identical.  They're called Type Ia
supernovae --- but never mind, they're basically identical explosions.
They're not simultaneous, but from the pale blue dot they look
simultaneous, because they arrive at the same time.  These red rims
represent the wavelength, which expands with the universe.  Now the
observed wavelengths for each supernova are identical in both models.
But in the Pinot Noir universe, the earlier explosion is older and
further away.  So though it takes up more space on the animation, it
is observed as fainter.  It's the brightness ratio of the supernovae
that tells us we live in an accelerating universe.

You can then work out what lensing does in an expanding universe.
We'll skip the animation and just quickly see some results.

\next

Basically, the difference in light travel times lets you calculate the
expansion time of the universe.  Here Jonathan Coles gets an estimate
of 13.5 billion years with an uncertainty of 15 percent.  Paraficz and
Hjorth plot the reciprocal of that quantity, the expansion rate.
Their age estimate is bit higher, but agreeing within uncertainties.

Which is very nice, but you'd really like to get those uncertainties
down.  You want ten times as many lenses, or a hundred times if you
can have them.  Is that possible?

\next

Here's a map of the sky with the known lenses shown on it.  This kind
of map is just a way of showing the whole sky at the same time.

\next

It's like this map, but for the sky instead of the globe.

\next

That's the Milky Way, which is beautiful to look at but makes it
difficult to see behind it, so not many lenses found there.  But
basically, there are lensing galaxies scattered all over the sky, you
just have to take long exposures and look through them very carefully.

Two big deep-sky surveys DES and Pan-Starrs that are starting up now,
would have thousands of lenses.  And the Large Synoptic Survey
Telescope, planned for the next decade, should see tens of thousands
of lenses.  And experiments with software searching show that
computers can find the easy ones, but are not nearly as good as a
trained human.  It really needs a human with intuition looking through
maybe a thousand big screenfuls to find one.

Five or six years ago, Kevin Schawinski, now our colleague at ETH but
then at Oxford, and Chris Lintott, did a crazy experiment.  They
appealed to members of the public to help classify galaxy images.  Now
Chris Lintott is also a broadcaster, and I guess he carefully picked a
slow news day to launch Galaxy Zoo, and somehow it went into the news
at prime time.  You can guess the rest of the story.  But there's
second part to the story, which is that with time the volunteers of
Zooniverse started asking for more difficult things.

\next

Here's one of several projects on now, which is transcribing
manuscript fragments written in ancient Greek.  There will be a
LensZoo in a few months, so watch this space.

Citizen science projects like zooniverse are not scientific
heavyweights yet, but they are getting increasingly important.  Some
enthusiasts are saying they will change the way science is done, in
the direction the University of Zuerich pioneered how most
universities are run.

In astronomy, amateurs have always played a role.  Could they start
playing a leading role in the future?  I don't know, but to conclude
today, I'd like to show you something really experimental.  Thanks to
Rafael Kueng for letting me be the first to present it.  It's platform
for modelling gravitational lenses, where someone could start at the
zooniverse level, but it's all open development, so if they want to
get their hands dirty like a full-time researcher, they can do that
too.

\next

The idea here is to sketch the arrival-time surface, identifying
maxima, minima, saddle points.  The machine will then take that sketch
as input to make a model of the mass distribution of the lens.  Then
we can see the model and see if it looks reasonable.

\bye
